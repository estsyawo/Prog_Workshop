\documentclass[12pt,a4paper]{article}
\usepackage[left=0.8in,top=1in,bottom=1in]{geometry}
\usepackage[latin1]{inputenc}
\usepackage{amsmath}
\usepackage{amsfonts}
\usepackage{amssymb}
\usepackage{hyperref}
\usepackage{graphicx}
\usepackage{bm}
\usepackage{enumitem}
\usepackage{booktabs}
\usepackage{multirow}
%\usepackage{natbib}
\usepackage[bibstyle=numeric, citestyle=authoryear, natbib=true,doi=false, url=true, backend=biber, maxbibnames=6, maxcitenames=4, uniquelist=false, uniquename=false, sorting=nyt]{biblatex}
\addbibresource{Paper.bib}

\begin{document}
	\title{R Programming for Economics \& Statistics}
	\author{Emmanuel Selorm Tsyawo\footnote{email: estsyawo@temple.edu} \\ \scriptsize Temple University, Department of Economics}
	\date{\today}
	\maketitle

	R is the most comprehensive statistical analysis package available. It incorporates all of the standard statistical tests, models, and analyses, as well as providing a comprehensive language for managing and manipulating data. New technology and ideas often appear first in R. \\
\textit{- Source: \url{http://analyticstrainings.com/?p=101}} 
	\section*{Why learn R ?}

	\begin{enumerate}[label=(\alph*)]
		\item R is free and open source; reviewed by many renowned international statisticians and computational scientists.
		
		\item A lot of documentation on-line; someone has already faced the problem you are facing and the solution is out there online.
		
		\item Greater flexibility in programming; you can avoid constraints with syntax-based programming languages.
		
		\item Numerous contributors of packages all over the world - an exponential growth in functionalities and accessibility
		
		\item Fast and lean on memory; R requires little memory to run because you only load packages you need. This allows R to execute codes rapidly.
		
		\item New books for R are constantly emerging. eg. Springer Use R! series, so you never lack adequate resources. 
		
	\end{enumerate}
The workshop comprises three sessions, each per meeting.

	\section{Session I}
	\subsection{Basics of R}
	The basics of R are required for creating and handling data objects and building an R programme.\\
	\textit{Reference: \cite[Chapters 2,3 \& 4]{bloomfield2014using}}
	\begin{enumerate}[label=(\alph*)]
		\item Creating, indexing, sub-setting and operating vectors
		\item Creating, indexing, sub-setting and operating matrices
		\item Creating, indexing, sub-setting and operating lists
		\item Handling strings and lists
		\item Loading data into R
		\item Data manipulation in R
		\item Plotting data in R
		\item Logicals (True/False checks)
		\item If/else statements 
		\item Loops - while loop, for loop, 
		\item Writing functions in R
		\item Generating random numbers and handling distributions/densities
	\end{enumerate}
	
	\subsection{Econometric models using R packages}
	R has a number of packages that enable the estimation of many econometric and statistical models. In this section, we explore functions in available packages for econometric and statistical models. Recall, a key here is the interpretability of results and not mere estimation. 
	\\ (Reference: \cite[Chapters 3, 5, \& 7]{kleiber2008applied} and \cite[Chapter 12]{wooldridge2010econometric})
	
	\begin{enumerate}[label=(\alph*)]
		\item Installing and using functions in R packages
		\item Ordinary least squares
		\item Logit/Probit
		\item Regression Models for Count Data 
		\item Quantile regression
		\item The Bootstrap
	\end{enumerate}
	
	\section{Session II}

	\subsection{Numerical Methods and Optimisation}
	This section serves to illustrate some functionalities in R for solving mathematical problems numerically. The main focus is on solving non-linear problems where analytical solutions are infeasible. \\(Reference: \cite{bloomfield2014using})
	
	\begin{enumerate}[label=(\alph*)]
		\item Solving a system of linear equations
		\item Zeros of a function
		\item Non-linear systems of equations
		\item Numerical differentiation and integration
		\item Unconstrained optimisation - one- \& multi-dimensional optimisation
		\item Constrained optimisation - one- \& multi-dimensional optimisation with equality and non-negativity constraints
	\end{enumerate}
	
	\section{Session III}
	\subsection{Programming your own model}
	In this last section, we illustrate, via examples, the programming of one's own model when there are unavailable functionalities. We also consider some machine learning models (if time permits).\\
	(References: Own codes)
	
	\begin{enumerate}[label=(\alph*)]
		\item OLS with heteroskedasticity-robust standard errors
		\item Writing your own maximum likelihood
		\item Quantile regression via the asymmetric Laplace density likelihood
		\item $ \ell_1 $ penalisation of the objective function, - the Lasso
	\end{enumerate}
	
	\section*{Useful references for further practice}
	\begin{enumerate}[label=(\alph*)]
		\item \cite{croissant2018panel} - a very good text on panel data models in R
		\item \cite{james2013introduction} - This text is a good introduction to statistical/machine learning
		\item  \cite{albert2009bayesian} - Are you a believer in the bayesian paradigm? Get on board!
		\item \cite{shumway2006time} - This book strikes a beautiful balance between the theory of Time series analysis and programming time series models in R.
	\end{enumerate}
	
	\section*{Downloads}
	The following software applications are required for the workshop. Download and install them for the workshop. Also, check for the compatibility with the operating system.
	\begin{itemize}
		\item \url{https://cran.r-project.org/bin/windows/base/} - Download and install R
		\item \url{https://www.rstudio.com/products/rstudio/download/} - Download and install Rstudio
	\end{itemize}
\end{document}